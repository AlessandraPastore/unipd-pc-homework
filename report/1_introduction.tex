\section{Introduction}\label{introduction}

% introduzione generica all'algoritmo e cosa fa
Given a complete directed  graph, Floyd-Warshall's algorithm, described in \cref{alg:sequential}, can be used to compute the shortest paths from all vertices to all vertices.
When applying the sequential version of the algorithm, the solution is reached with a complexity of \(O(n^3)\), with \(n\) as number of vertices.
Unlike Dijkstra's algorithm for the shortest path problem, Floyd-Warshall's algorithm can also be used in case of negative costs and can identify the presence of negative cycles 
(the correctness of the algorithm is not guaranteed in such cases)~\cite{fischetti}.

The algorithm takes as input an adjacency matrix of $n \times n$  distances $c[i,j]$ (with $i,j \in V$ and $n$ number of vertices) representing an oriented graph, and as output it returns two matrices:
\begin{itemize}
    \item a distance cost matrix where \(d[i,j]\) is the cost of the shortest path from \(i\) to \(j\)
    \item a predecessor matrix where \(pred[i,j]\) is the predecessor of \(j\) in the shortest path from \(i\) to \(j\)
  \end{itemize}

The overall algorithm works by testing every intermediate vertex \(h\) to find the shortest path from a vertex \(i\) to a vertex \(j\),
comparing the new path cost with the current one (best so far) as described by \cref{eq:cost}~\cite{fischetti}.

\begin{equation} \label{eq:cost}
    d[i,j] = \min\left\{ d[i,j], d[i,h]+d[h,j] \right\}
\end{equation}

Once the algorithm has been run, we can use \(pred\) to navigate from one node to another via the best route.

\begin{figure}[htbp]
    \centering
    \begin{minipage}{.5\textwidth}
        \begin{algorithm}[H]
            \SetKwFunction{FStop}{Stop}
    \For{\(i \in [1{\twodots}n]\)}{
        \For{\(j \in [1{\twodots}n]\)}{
            \( d [ i , j ] \leftarrow c [ i , j ] \)

            \( pred [ i , j ]  \leftarrow i \)
        }
    }
    \For{\(h\in [1{\twodots}n]\)} { \label{alg:for-h}
        \For{\(i \in [1{\twodots}n]\)} { \label{alg:for-i}
            \For{\(j \in [1{\twodots}n]\)}{ \label{alg:for-j}
                \If{\( d [ i , h ] + d [ h , j ] < d [ i , j ] \)} {
                    \( d [ i , j ] \leftarrow d [ i , h ] + d [ h , j ] \)

                    \( pred [ i , j ] \leftarrow pred [ h , j ] \)
                }
            }
        }
        \For{\(i \in [1{\twodots}n]\)} {
            \If{\(d[i,i]< 0\)} {
                \FStop{“negative cycles”}
            }
        }
    }
\caption{Floyd-Warshall's sequential algorithm.}
\label{alg:sequential}
\end{algorithm}
    \end{minipage}
\end{figure}

\begin{figure}[htbp]
    \begin{subfigure}[b]{0.19\textwidth}
        \centering
        \includegraphics[width=\textwidth]{media/example_floyd_warshall}
        \caption{Input graph}
        \label{fig:example-floyd-warshall-input-graph}
    \end{subfigure}
    \begin{subfigure}[b]{0.28\textwidth}
        \centering
        {\renewcommand{\arraystretch}{1.25}%
        \setlength{\tabcolsep}{0.5em} % for the horizontal padding
        \begin{tabular} { c | c | c | c | c | c | }
            \mc{} & \mc{\(1\)} & \mc{\(2\)} & \mc{\(3\)} & \mc{\(4\)} & \mc{\(5\)} \\ \cline{2-6}
        \(1\) & \(0\) & \(3\) & 8 & \( \infty \) & \( - 4 \) \\  \cline{2-6}
        \(2\) & \( \infty \) & \(0\) & \( \infty \) & \(1\) & \(7\)   \\ \cline{2-6}
        \(3\) & \( \infty \) & \(4\) & \(0\) & \( \infty \) & \( \infty \) \\ \cline{2-6}
        \(4\) & \(2\) & \( \infty \) & \( - 5 \) & \(0\) & \( \infty \) \\ \cline{2-6}
        \(5\) & \( \infty \) & \( \infty \) & \( \infty \) & \(6\) & \(0\) \\ \cline{2-6}
        \end{tabular}}
        \caption{Input matrix}
        \label{fig:example-floyd-warshall-input-matrix}
    \end{subfigure}
    \begin{subfigure}[b]{0.26\textwidth}
        \centering
        {\renewcommand{\arraystretch}{1.25}%
        \setlength{\tabcolsep}{0.5em} % for the horizontal padding
        \begin{tabular} { c | c | c | c | c | c | }
            \mc{} & \mc{\(1\)} & \mc{\(2\)} & \mc{\(3\)} & \mc{\(4\)} & \mc{\(5\)} \\ \cline{2-6}
            \(1\)& \(0\) &\(1\) &\(-3\) &\(2\) &\(-5\)\\  \cline{2-6} \(2\)&\(3\) &\(0\) & \(-4\)&\(1\) &\(-1\)\\  \cline{2-6} \(3\)&\(7\) &\(5\) &\(0\) &\(5\) &\(3\)\\  \cline{2-6} \(5\)&\(2\) &\(-1\) &\(-5\) &\(0\) &\(-2\)\\  \cline{2-6} \(5\)&\(8\) &\(5\) &\(1\) &\(6\) &\(0\) \\ \cline{2-6}
            \end{tabular}}
        \caption{Output: distance matrix}
        \label{fig:xample-floyd-warshall-output-matrix-d}
    \end{subfigure}
    \begin{subfigure}[b]{0.25\textwidth}
        \centering
        {\renewcommand{\arraystretch}{1.25}%
        \setlength{\tabcolsep}{0.5em} % for the horizontal padding
        \begin{tabular} { c | c | c | c | c | c | }
            \mc{} & \mc{\(1\)} & \mc{\(2\)} & \mc{\(3\)} & \mc{\(4\)} & \mc{\(5\)} \\ \cline{2-6}
            \(1\)&\(0\) &\(2\) &\(3\) &\(4\) &\(0\)\\  \cline{2-6} \(2\)&\(3\) &\(1\) &\(3\) &\(1\) &\(0\)\\  \cline{2-6} \(3\)&\(3\) &\(2\) &\(2\) &\(1\) &\(0\)\\  \cline{2-6} \(4\)&\(3\) &\(2\) &\(3\) &\(3\) &\(0\)\\  \cline{2-6} \(5\)&\(3\) &\(2\) &\(3\) &\(4\) &\(4\) \\ \cline{2-6}
        \end{tabular}}
        \caption{Output: \(pred\) matrix}
        \label{fig:xample-floyd-warshall-output-matrix-pred}
    \end{subfigure}
    \caption{Example of the input and the output for the execution of the Floyd-Warshall algorithm.}
    \label{fig:example-floyd-warshall}
\end{figure}

%esempio?

% altro esempio?

% performance dell'algoritmo?



\FloatBarrier
